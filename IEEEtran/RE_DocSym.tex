
\documentclass[conference]{IEEEtran}

\ifCLASSINFOpdf
   \usepackage[pdftex]{graphicx}
  % declare the path(s) where your graphic files are
   \graphicspath{{C:/Users/chuprina/Documents/PhD/Conferences/RE/2018/IEEEtran/pictures/}}
  % and their extensions so you won't have to specify these with
  % every instance of \includegraphics
  \DeclareGraphicsExtensions{.pdf,.jpeg,.png}
\else
\fi

\usepackage{cite}
\usepackage{amsmath,amssymb,amsfonts}
\usepackage{algorithmic}
\usepackage{graphicx}
\usepackage{textcomp}
\usepackage{xspace}
\usepackage{url}
\def\BibTeX{{\rm B\kern-.05em{\sc i\kern-.025em b}\kern-.08em
    T\kern-.1667em\lower.7ex\hbox{E}\kern-.125emX}}
    
% Put edit comments in a really ugly standout display
\usepackage{times}
 
\newcommand{\todo}[1]{\textbf{\textcolor{red}{TODO: #1}}} 
% Macros for proof-reading & corrections
\usepackage[normalem]{ulem} % for \sout  
\usepackage{xcolor} 

\usepackage[utf8]{inputenc}

\usepackage{hyperref}
\usepackage{url}

\hyphenation{op-tical net-works semi-conduc-tor}
 
\newcommand{\af}{\textsc{AF3 }}
\newcommand{\autof}{\textsc{AutoFOCUS 3 }}
\newcommand{\asp}{\textit{aspects }} 
\newcommand{\cbc}{{Categorization-based Concept }}
\newcommand{\care}{{Categorization-based Approach }}
\newcommand{\cc}{\textsc{CBC }}
\newcommand{\ca}{\textsc{CARE }}

\usepackage{ifthen}
\usepackage{amssymb} 
\newboolean{showcomments}  
\setboolean{showcomments}{true} % toggle to show or hide comments
\ifthenelse{\boolean{showcomments}}     
  {\newcommand{\nb}[2]{
    \fcolorbox{gray}{yellow}{\bfseries\sffamily\scriptsize#1}
    {$\blacktriangleright$#2$\blacktriangleleft$}
   }
   \newcommand{\version}{\emph{\scriptsize$-$working$-$}} 
  }
  {\newcommand{\nb}[2]{}
   \newcommand{\version}{} 
  } 
\newcommand\tatiana[1]{\nb{Tatiana}{\textcolor{blue}{#1}}}
\newcommand\levi[1]{\nb{Levi}{\textcolor{teal}{#1}}}
\newcommand\hernan[1]{\nb{Hernan}{\textcolor{teal}{#1}}}


\begin{document}

\title{Categorization-based Approach in Requirements Engineering.} 


% author names and affiliations
% use a multiple column layout for up to three different
% affiliations
\author{\IEEEauthorblockN{Tatiana Chuprina}
\IEEEauthorblockA{Model-based System Engineering\\
Fortiss GmbH\\
Munich, Germany\\
email: chuprina@fortiss.org}}
%\and
%\IEEEauthorblockN{Homer Simpson}
%\IEEEauthorblockA{Twentieth Century Fox\\
%Springfield, USA\\
%Email: homer@thesimpsons.com}
%\and
%\IEEEauthorblockN{James Kirk\\ and Montgomery Scott}
%\IEEEauthorblockA{Starfleet Academy\\
%San Francisco, California 96678--2391\\
%Telephone: (800) 555--1212\\
%Fax: (888) 555--1212}}

% conference papers do not typically use \thanks and this command
% is locked out in conference mode. If really needed, such as for
% the acknowledgment of grants, issue a \IEEEoverridecommandlockouts
% after \documentclass

% for over three affiliations, or if they all won't fit within the width
% of the page, use this alternative format:
% 
%\author{\IEEEauthorblockN{Michael Shell\IEEEauthorrefmark{1},
%Homer Simpson\IEEEauthorrefmark{2},
%James Kirk\IEEEauthorrefmark{3}, 
%Montgomery Scott\IEEEauthorrefmark{3} and
%Eldon Tyrell\IEEEauthorrefmark{4}}
%\IEEEauthorblockA{\IEEEauthorrefmark{1}School of Electrical and Computer Engineering\\
%Georgia Institute of Technology,
%Atlanta, Georgia 30332--0250\\ Email: see http://www.michaelshell.org/contact.html}
%\IEEEauthorblockA{\IEEEauthorrefmark{2}Twentieth Century Fox, Springfield, USA\\
%Email: homer@thesimpsons.com}
%\IEEEauthorblockA{\IEEEauthorrefmark{3}Starfleet Academy, San Francisco, California 96678-2391\\
%Telephone: (800) 555--1212, Fax: (888) 555--1212}
%\IEEEauthorblockA{\IEEEauthorrefmark{4}Tyrell Inc., 123 Replicant Street, Los Angeles, California 90210--4321}}




% use for special paper notices
%\IEEEspecialpapernotice{(Invited Paper)}




% make the title area
\maketitle

% As a general rule, do not put math, special symbols or citations
% in the abstract
\begin{abstract}
\tatiana{Here we present \care (CARE) for handling unstructured 
textual requirements in requirements engineering process. \ca method brings unformalized,
 vague textual requirements into a structured form applying general categorization to them; 
so that, requirements engineer (RE) proceeds with text of the requirements without re-writing 
them into formal sentences. In other words, \ca doesn't force RE to formalize requirements 
with help of rigid patterns or formal languages and lets a freedom in creativity 
with elaboration of textual requirements. \ca organize requirements, written in natural language, 
into a structure based on their categories (functional and non-functional).}

\tatiana{Requirements consists of incoming information to all phases of system development 
(design, testing and others). We can consider these inputs as meta-data of the requirements. 
That means, \ca structures requirements' meta-data by categories building a ``framework'' 
of textual requirements. This framework with the input-data constitutes a \textit{new artifact} 
in system development process. It represents a requirements as meta-model attached 
to the textual requirements.}

\tatiana{Moreover, because requirements meta-data spread throughout all system development levels, 
all stakeholders in system development life-cycle can employ \care to contribute 
into requirements engineering process. Consequently, we affirm, that \ca implies not only 
an approach to requirements categorization, but also supports a categorization-based process flow 
for all stakeholders. In this paper, we give an example of \ca usage in requirements review.}

\tatiana{Summing up all points from above, we postulate, that \care increases quality of textual requirements 
categorizing them. Additionally, \ca impacts efficiently on requirements engineering process 
in system development supporting more precise requirements analysis.\\}


In this paper \cbc (CBC) is presented, that proposes a solution for handling
unstructured textual system requirements. The \cc offers a method for
requirements elaboration and for their systematic review. We claim, that \cbc
provides a common instrument for all roles of system development process
exposing a new Artifact. Thus, the research of \cc raises the scientific
questions regarding quality of system requirements.

Furthermore, this article considers an implementation of \cbc in \autof (AF3)
development tool~\cite{AF} as a framework called ``\textit{aspects}''. Here \asp
represent requirements' categories, which apply to requirements. They constitute
the templates inferred from requirements facets.

\end{abstract}

% no keywords




% For peer review papers, you can put extra information on the cover
% page as needed:
% \ifCLASSOPTIONpeerreview
% \begin{center} \bfseries EDICS Category: 3-BBND \end{center}
% \fi
%
% For peerreview papers, this IEEEtran command inserts a page break and
% creates the second title. It will be ignored for other modes.
\IEEEpeerreviewmaketitle

\section{Introduction}
\label{sec:Intro} 
Initial phase of system development process (Requirements engineering) orchestrates a whole work flow of the development. It initiates with collecting customers' requirements, which are objectives for a future system; it continues with their  analyses and transformations into high level and low level requirements, which are inputs for system design process; and finally, it goes with testing the system design against the requirements. It means, that requirements serve as an income to a system development and influence the result of this process; apparently, their quality correlate with the quality of the end product [].

Still, the question about measuring quality of requirements remains problematical due to its subjectivity[].
\tatiana{discover a problem here}

 Here we present our approach to the quality measurement suggesting its mathematical explanation. Comparering with existing approaches, discussed in section , ours (title) considers a relation between quality of requirements and the resulted product measuring time and resources consumption for requirements engineering phase and further software development.

quality of requirements

resulted product

time and resources consumption 
\inputencoding{utf8}
\section{Categorization-based Concept}
\label{sec:cbc} 

Usually, requirements provided by stakeholders have poor structure and ambiguous description of a designed system. This can possibly lead to misunderstandings from developers or reviewers. Furthermore, a poor readability of requirements might result in mistakes within system design or time increasing for their review. All these points indicate a low quality. 

In order to improve quality of requirements and prevent mentioned problems, requirements should be structured and formalized to become more exact. Here we claim, that \cbc aims this purpose providing a method for arranging requirements in a consistent structure.

The idea behind \cc arises from common requirements categorization and goes ahead forming a structure of textual requirements. Requirements are categorized by their characteristics. Therefore, one or more categories can be attached to a single requirement, depending on the requirement's composition. The number of applied categories increases with rising complexity of the considered requirement. That implies, a requirement with a range of categories should be splitted into several simplified requirements with less number of attached categories, if it is possible with respect to context.

The user can define, which categories correspond to each requirement. Moreover, \cc provides a certain degree of freedom for the user to choose a granularity of every attached category. In other words, the user makes a decision on how detailed the requirements should be considered and categorized. After an analysis of the whole scope of requirements, every requirement should be categorized.  

The output of this procedure is structured and more consistent requirements with attached set of corresponding categories. This set constitutes a new artifact in system development process, that represents a full structure of the requirements. Such structured requirements are more precise and unambiguous. Thanks to that the reviewing process becomes easier and more accurate; risk of mistakes in system design declines due to better legibility of requirements. We can conclude, that quality of requirements increases by applying \cbc.

Additionally, the idea of the \cc is general and allows a user to be independent from specific tools. It can be applied within already known for a user chain of tools and thereby, save costs and time for a project realization, instead of training employees for new extra tools. 

\subsection{\cbc in \autof}
\label{sec:cbcinaf}

\cc has been implemented in \autof. In \af requirements categorization incarnates so called \asp (templates for requirements characteristics, which represent requirements' categories).   

In \af eight \asp have been created, which are depicted in \autoref{fig:fig_aspects}. 

\begin{itemize}
\item	\textit{Timing aspect} - the requirement refers to timing constraints
\item	\textit{Signal aspect} - the requirement defines a signal, its type and its range
\item	\textit{Parameter definition aspect} - the requirement defines a parametric value. This allows to re-use the requirement by changing the value of the parameter
\item\textit{Safety level aspect} - the requirement defines a safety level of the system
\item	\textit{Mode aspect} - the requirement defines an operational state of system e.g. active/inactive mode.
\item	\textit{Design choice aspect} - the requirement was derived from a design choice of the software developer
\item	\textit{Functional aspect}- the requirement specifies a particular behavior by relating inputs and outputs
\item	\textit{Temporal property aspect} - the requirement defines a property expressed by temporal patterns
\end{itemize}

\begin{figure}[!t]
\centering
\includegraphics[width=2in]{asset_aspects.png}
\caption{List of \textit{aspects} in \autof}
\label{fig:fig_aspects}
\end{figure}

However, the number of \asp can vary with respect to a context of project and needs of requirements engineer/ reviewer. \textit{Aspects} can be applied to system requirements from initial stage of the development process, from requirements elicitation. \textit{Aspects} form a structure of the analyzed requirements tagging them. \autoref{fig:pd_aspect_tag} presents a view of a textual requirement with assigned \textit{Parameter definition aspect}.
\begin{figure}[!t]
\centering
\includegraphics[width=1.6in]{param_aspect_tag.png}
\caption{Requirement with Assigned \textit{Parameter definition aspect} in \autof}
\label{fig:pd_aspect_tag}
\end{figure}

As it was mentioned above, every aspect provides a template for requirement's main characteristics. The templates collect meta-data of requirements. For instance, \autoref{fig:pd_aspect_view} shows a view of \textit{Parameter definition aspect} template attached to a corresponding requirement. 
\begin{figure}[!t]
\centering
\includegraphics[width=1.6in]{aspect_view.png}
\caption{\textit{Parameter definition aspect} template in \autof}
\label{fig:pd_aspect_view}
\end{figure}

As it is shown in \autoref{fig:pd_aspect_view}, the aspect template includes information from tagged requirement, thereby binding the aspect and the requirement together. Data captured in aspects' templates depends on the granularity of the applied category (aspect), chosen by the \cc user.  Furthermore, the structured requirements with assigned \asp appear as an input for development and reviewing processes. This means, that \asp link artifacts of these processes (e.g system architectures, low-level requirements (LLR), review results, etc.) supporting traceability between them; contribute in mapping requirements into system design. 

In addition, it is easier to check structured requirements.

Summing up all statements from above, \asp based on \cbc, facilitate in increasing quality of requirements. 

Let us consider \cc with the ”Router” use case, which was inspired by an industrial project being conducted together with Airbus and Fortiss. 

\subsection{Use Case for applying \cbc}
\label{sec:usecase} 

``Router'' use case exposes an example of a simple ``Router'' component with configured quality of service (QoS) management, which is used in telecommunication networks. In this use case Stakeholders have provided the system requirements (\autoref{tab:SR}) as a text document with semi-structured information and a simple component diagram (\autoref{fig:route}). 
\begin{figure}[!t]
\centering
\includegraphics[width=3.5in]{use_case/router.png}
\caption{Diagram of the Router in the Network}
\label{fig:route}
\end{figure}

\begin{table}
	\centering
		\begin{tabular}{p{8cm}}
\small
\begin{center}
\textbf{\textit{System requirements.}}
\end{center}
``In communication system Data Link (DL) is usually used by several users. The users can proceed with different types of traffic. For better use of Data Link resources, traffic from diverse users can differ by priorities (depends on QoS and user's agreement). Depending on the priority, traffic should be forwarded along a certain route with a defined latency. For high priority packets the route shall be defined with the lowest latency. The lower priority of traffic, the higher latency the provided route has. Configuration input for Selector should be specified by QoS agreement of user.

Message Filter defines priority of packets and send them directly to Selector. 

Selector should:
\begin{itemize}
	\item receive packets from Message Filter;
  \item define a proper route, according to packet priority;
\end{itemize}

Packets shall be sent from Selector to Data Link using 3 routes.

All QoS policy is defined in Selector component. 

There are 3 data priorities: 
\begin{enumerate}
	\item High priority information - Selector should route such packets into 1st route only! 
  \item	Middle level priority - packets can be sent by Selector into 1st route , if the route is free; otherwise, traffic is transmitted into 2nd route;
  \item	Low priority - data can be sent in 3 routers, depends on which route is free at the moment.''
\end{enumerate}
\end{tabular}
	\caption{System Requirements Provided by Stakeholders}
	\label{tab:SR}
\end{table}

\normalsize 
\subsubsection{Requirements Structuring}
\label{sec:reqstruct}
Initial phase of development process is elicitation and analysis of requirements. We claim, that during analysis process \asp should be applied to the requirements in order to categorize them and form their structure. Considering further the ``Router'' use case, the following \asp can be assigned to the given requirements:\\

\small ``Req-1.	In communication system Data Link is usually used by several users.''

\normalsize[\textit{Mode aspect}] implies, that DL is in active state, when the users send traffic.\\

\small ``Req-2.	The users can proceed with different types of traffic.''

\normalsize [\textit{Signal aspect}] describes, that the packets have different types.\\

\small ``Req-3.	For better use of Data Link resources, traffic from diverse users can differ by priorities (depends on QoS in user's agreement).''

\normalsize [\textit{Signal aspect}] describes the traffic.\\

\small ``Req-4.	Depending on the priority, traffic should be forwarded along a certain route with a defined latency.'' 

\normalsize \begin{itemize}
	\item \textit{Parameter definition aspect} indicates, that priorities, routes and latency can be kept as re-useable parameters;
	\item \textit{Signal aspect} describes input for the routes;
	\item \textit{Functional aspect} describes the behavior of ``Packets Selector'';
	\item \textit{Temporal property aspect} shows, that the statement includes a condition;
\end{itemize}

The rest of the requirements should be analyzed and structured similarly.

After requirements have been categorized, the requirements engineer can proceed with the following inspection. First of all, he/she should check, if all requirements have \asp. Next, re-consider the requirements with more than 1 aspect in their categorization; it can mean, that such requirements are not precise enough. For example, the requirement ``Req-4'':\\

\small ``Req-4. Depending on the priority, traffic should be forwarded along a certain route with a defined latency.``

\normalsize
\begin{itemize}
	\item \textit{Parameter definition aspect}, 
	\item \textit{Signal aspect},
	\item \textit{Functional aspect}, 
	\item \textit{Temporal property aspect.}
\end{itemize}

It can be splitted into several requirements according to \asp categories:\\

For [\textit{Functional aspect}]:

\small
``Req-4.1.	 Selector should forward traffic along a certain route.''\\

\normalsize For [\textit{Parameter definition aspect}]:

\small
``Req-4.2.	 Every route operates with defined latency.''

``Req-4.3.	There are several (three) routes.''

``Req-4.4.	There are several priorities for traffic.''\\

\normalsize For [\textit{Signal aspect}]:

 \small ``Req-4.1.	Traffic is prioritized.''\\

\normalsize For [\textit{Temporal property aspect}]:

\small ``Req-4.2.	Traffic priority defines a route for traffic.''\\

\normalsize The ideal case is when every requirement has only one aspect, but in real project requirements categories usually overlap each other. Therefore, the number of \asp attached to a requirement should tend to be minimized.

After applying this method also to the other requirements, we have obtained the following structure of requirement, presented in \autoref{tab:structuredHLR}.

\begin{table*}
	\centering
		\begin{tabular}{| p{11cm}  | p{6cm} |}
			\small ``Req-1.	 In communication system Data Link is usually used by several users.'' &	[\textit{Mode aspect}]\\
``Req-2.	The users can proceed with different types of traffic.'' & [\textit{Signal aspect}]\\	
``Req-3.	For better use of Data Link resources, traffic from diverse users can differ by priorities (depends on QoS in user's agreement).'' & [\textit{Signal aspect}] \\
``Req-4.1	 Selector should forward traffic along a certain route.''& [\textit{Functional aspect}]\\ 
``Req-4.2	Every route operates with defined latency.'' & [\textit{Parameter definition aspect}]\\		
``Req-4.3	There are several (three) routes.'' & [\textit{Parameter definition aspect}] \\
``Req-4.4 There are several priorities for traffic.'' & [\textit{Parameter definition aspect}]\\
``Req-4.5	Traffic is prioritized.'' & [\textit{Signal aspect}] \\
``Req-4.6	Traffic priority defines a route for traffic.'' & [\textit{Temporal property aspect}] 	\\
``Req-5.	For high priority packets the route shall be defined with the lowest latency.'' & [\textit{Temporal property aspect}] \\
``Req-6.	The lower priority of traffic, the higher latency the provided route has.'' & [\textit{Temporal property aspect}] \\
``Req-7.	Configuration input for Selector should be specified by QoS agreement of user.'' & [\textit{Signal aspect}] \\
``Req-8.1	Message Filter defines priority of packets.'' & [\textit{Functional aspect}] \\	
``Req-8.2 Message Filter sends packets directly to Selector.'' &  [\textit{Functional aspect}] \\
``Req-9.1	Selector should receive packets from Message Filter.'' & [\textit{Functional aspect}] \\
``Req-9.2	Selector should define a proper route, according to packet priority.'' & [\textit{Functional aspect and Temporal property aspect}]\\
``Req-10.1	Selector shall send packets to Data Link.'' & [\textit{Functional aspect and Signal aspect}] \\
``Req-10.2	Selector shall use 3 routes for packets sending to Data Link.'' & [\textit{Functional aspect}] \\
``Req-10.3	There are 3 routes for sending packets from Selector to Data Link.'' & [\textit{Parameter definition aspect}] \\
``Req-11	All QoS policy is defined in Selector component.'' & [\textit{Functional aspect}] \\
``Req-12.1	There are 3 data priorities.'' & [\textit{Parameter definition aspect}] \\
``Req-12.2.	High priority information - Selector should route such packets into 1st route only!'' & 
[\textit{Temporal property aspect}] \\
``Req-12.3.	Middle level priority - packets can be sent by Selector into 1st route , if the route is free; otherwise, traffic is transmitted into 2nd route.'' & [\textit{Temporal property aspect}] \\
``Req-12.4.	Low priority - data can be sent in 3 routers, depends on which route is free at the moment.'' & [\textit{Temporal property aspect}] \\
		\end{tabular}
	\caption{Structured Requirements with Attached Aspects}
	\label{tab:structuredHLR}
\end{table*}

\normalsize
Now we have more consistent high-level requirements (HLR) in comparison with the requirements initially provided by the stakeholder. That means, the requirements are better structured, more precise and less complex. These characteristics rise quality of the requirements. Therefore, it is easier to work with such requirements during development phase or review process.

These structured HLR now can be an input for system design process. Developers can also analyze the requirements and provide additional requirements, which were inferred from HLR. For this purpose, there exists a \textit{Design Choice aspect}. This aspect indicates, that changes in requirements have been done by developers team. Example of such requirement is provided:

\small''Req-13.  Packets follow the Ethernet standard.''

\normalsize the requirement now has two \asp [\textit{Signal aspect} and \textit{\textbf{Design Choice aspect}}]. 


\normalsize \subsubsection{Requirements Review}
\textit{Aspects} support also requirements review process. Every \textit{aspect} provides a check-list specified for respective category of requirement. This feature serves for more rigorous review of the requirements instead of a general check. More over, different \asp imply different activities of requirements checking. As an example from the use case, to inspect Req-10.2: \\

\small ``Req-10.2.  Selector shall use 3 routes for packets sending to Data Link.'' 

\normalsize [\textit{Functional aspect}], here reviewer checks correctness of algorithms implemented within this requirement. However, the same check is useless for Req-4.3: \\

\small ``Req-4.3.  There are several (three) routes.'' 

\normalsize ,which is defined by [\textit{Parameter definition aspect}].\\

Working with HLR structured by \asp, the reviewer can trigger the following revision:
\begin{itemize}
	\item whether \textit{aspects} have been applied to each requirement,
  \item whether the \textit{aspects} match appropriately with the considered requirements (the reviewer can propose additional \asp or make changes in the ones already attached),
	\item whether all \asp check-lists give successful outcomes.
\end{itemize}

%\input{review}
\section{Evaluation and Validation}
\label{eval}
Firstly, \cbc has been validated by industry use case within ASSET project during avionics component development. 
%The outcames of this  – what impact the Aspects make on requirements reviewing process – handle requirements complexity.

Next step for evaluation and validation of \cbc can be series of empirical studies conducting with students, at Fortiss research institute to find out, how a user can handle unstructured system requirements applying \cbc and without \cc. The outcomes will be compared in order to check, whether quality of the considered requirements increases with \cc application. For this purpose criteria of comparison can be defined as:
\begin{itemize}
	\item time-consumption during system development based on the structured and unstructured requirements
	\item comprehension of the structured and unstructured requirements (e.g. time for read and understanding requirements)
	\item number and types of errors have been corrected within reviewing process
\end{itemize}

However, these criteria can vary depending on further results of study.

\section{Related Work}
\label{sec:relatedwork} 

\subsection{what's quality means?}

Despite on multiple publications about requirements quality and its assessment, the ``quality'' as a term is still subjective.~\cite{Mund:2017},~\cite{Femmer:2017}.
Industry standards [ISO/IEC/IEEE:29148:201] prescribe characteristics and criteria, which presumed effective for improving quality of requirements e.g completeness, unambiguity for sets of requirements and difficulty, dependency for a single requirement. Moreover, Lamsweerde provides possible defect-based checklist~\cite{Lamsweerde:2009}.
Now exists a vast majority of quality definitions:
\begin{enumerate}
	\item in industry Standards - set of attributes such as completeness, unambiguity... [ISO/IEC/IEEE-29148];
	\item in scientific point of view, [Lamsweerde] gives a general list of characteristics, [Pohl] proposed a framework defining dimensions of quality (• the specification dimension, • the representation dimension, the agreement dimension.) ;
	\item check requirements language for lacking of errors, defects, ambiguity and possible reasons for incomprehension [];
	\item different kind of quality models such as activity-based[10], natural-language requirements specifications[7] %(activity, stupid!)
\end{enumerate}


\subsection{Metrics for quality measure}
\begin{itemize}
	\item methodology for context-specific RE artifact quality measuring has in its base an activity-based quality model [activity, stupid!] - brief description
	\item the mentioned attributes should be satisfied 
	\item some researchers shift their look to product quality measurement e.g [measuring success], however 
\end{itemize}

All these metrics intend to intrinsic inspection of requirements. In contrast to them, we propose the quantitative metrics for assessment of requirements quality
%\section{ExpectedContributions}
Conducting this research the following goals will be achieved:
\begin{itemize}
\item	to bridge a gap between requirements engineers and reviewers providing to them a feasible method based on requirements \textit{aspects}, which guides both roles within requirements reviewing process and supports a communication flow between these two roles. That leads to ease of redundancy in the work-flow,
\item	to close a gap between requirements reviewer and system developers by providing a structured constitution of requirements which is in help to concentrate a precise overview on the requirements (as on a whole scope of them so as only on specific category of requirements or part of them, depending on categorization granularity). As a result, a reviewer can provide a more specified and detailed response to developers on any changes in a system,
\item	moreover, \textit{aspects} concept links a bridge between requirements engineers and developers, so that a structure of \textit{aspects} itself includes key “hints” for every requirements category, which help in seamlessly mapping requirements into design of a system,
\item	the \textit{aspects} is a general concept, which can be applied independently on a specific tool chain. They can be simply applied with a familiar development tool. It gives a flexibility in use and reduces the cost for new tools learning,
\item	semi-automatically check-lists, which are an extensions for every aspect, useful for making a requirements review easier and, consequently, less time- and resource-consuming.
\end{itemize}
All in all, the \textit{aspects} concept provides a common instrument for all mentioned above roles (requirements engineers, reviewers, developers). All these points can increase a quality of requirements and thereby, save a significant amount of resources in system development such as project time and expenses.

\section{Current Status}
At the moment, the research goes further with the idea of \cbc. It implies to consider an impact of \cc on a whole development process including testing phase. Also the study is supposed to cover requirements reuse and a role of the \cbc in supporting that process.

Currently, question of relevant and sufficient methods for proper evaluation of the \cbc still arises. The investigation of possible techniques is in process.


%considers relevant and sufficient methods for proper evaluation of the \cbc. Simultaneously to this research task, further implementation of the \asp concept in \af tool and its testing continues. Moreover, the study is supposed to consider an impact of the \cbc on other phases of development process, such as system testing, requirements reuse and etc.

%Next steps within the research appear as a checking of usability of the concept conducting a series of empirical experiments.


%Implemented within the MIRA framework \cite{18MIRA} in \autof and reflecting requirements engineering activities form the initial stage of a development process, the \textit{aspects} can represent a general concept supporting its unique method, which may be applied within other tools thereby, providing tool\textbackslash platform- independence. The next pace in this direction of the research is a deeper investigation of this theory and its consideration on practice.
%
%And the most challenging question about proper and adequate validation technics for the Aspects concept will be considered in future work in this study.

% no \IEEEPARstart


% You must have at least 2 lines in the paragraph with the drop letter
% (should never be an issue)


%\hfill 
% 
%\hfill 
%
%\subsection{Subsection Heading Here}
%Subsection text here.
%
%
%\subsubsection{Subsubsection Heading Here}
%Subsubsection text here.


% An example of a floating figure using the graphicx package.
% Note that \label must occur AFTER (or within) \caption.
% For figures, \caption should occur after the \includegraphics.
% Note that IEEEtran v1.7 and later has special internal code that
% is designed to preserve the operation of \label within \caption
% even when the captionsoff option is in effect. However, because
% of issues like this, it may be the safest practice to put all your
% \label just after \caption rather than within \caption{}.
%
% Reminder: the "draftcls" or "draftclsnofoot", not "draft", class
% option should be used if it is desired that the figures are to be
% displayed while in draft mode.
%
%\begin{figure}[!t]
%\centering
%\includegraphics[width=2.5in]{myfigure}
% where an .eps filename suffix will be assumed under latex, 
% and a .pdf suffix will be assumed for pdflatex; or what has been declared
% via \DeclareGraphicsExtensions.
%\caption{Simulation results for the network.}
%\label{fig_sim}
%\end{figure}

% Note that the IEEE typically puts floats only at the top, even when this
% results in a large percentage of a column being occupied by floats.


% An example of a double column floating figure using two subfigures.
% (The subfig.sty package must be loaded for this to work.)
% The subfigure \label commands are set within each subfloat command,
% and the \label for the overall figure must come after \caption.
% \hfil is used as a separator to get equal spacing.
% Watch out that the combined width of all the subfigures on a 
% line do not exceed the text width or a line break will occur.
%
%\begin{figure*}[!t]
%\centering
%\subfloat[Case I]{\includegraphics[width=2.5in]{box}%
%\label{fig_first_case}}
%\hfil
%\subfloat[Case II]{\includegraphics[width=2.5in]{box}%
%\label{fig_second_case}}
%\caption{Simulation results for the network.}
%\label{fig_sim}
%\end{figure*}
%
% Note that often IEEE papers with subfigures do not employ subfigure
% captions (using the optional argument to \subfloat[]), but instead will
% reference/describe all of them (a), (b), etc., within the main caption.
% Be aware that for subfig.sty to generate the (a), (b), etc., subfigure
% labels, the optional argument to \subfloat must be present. If a
% subcaption is not desired, just leave its contents blank,
% e.g., \subfloat[].


% An example of a floating table. Note that, for IEEE style tables, the
% \caption command should come BEFORE the table and, given that table
% captions serve much like titles, are usually capitalized except for words
% such as a, an, and, as, at, but, by, for, in, nor, of, on, or, the, to
% and up, which are usually not capitalized unless they are the first or
% last word of the caption. Table text will default to \footnotesize as
% the IEEE normally uses this smaller font for tables.
% The \label must come after \caption as always.
%
%\begin{table}[!t]
%% increase table row spacing, adjust to taste
%\renewcommand{\arraystretch}{1.3}
% if using array.sty, it might be a good idea to tweak the value of
% \extrarowheight as needed to properly center the text within the cells
%\caption{An Example of a Table}
%\label{table_example}
%\centering
%% Some packages, such as MDW tools, offer better commands for making tables
%% than the plain LaTeX2e tabular which is used here.
%\begin{tabular}{|c||c|}
%\hline
%One & Two\\
%\hline
%Three & Four\\
%\hline
%\end{tabular}
%\end{table}


% Note that the IEEE does not put floats in the very first column
% - or typically anywhere on the first page for that matter. Also,
% in-text middle ("here") positioning is typically not used, but it
% is allowed and encouraged for Computer Society conferences (but
% not Computer Society journals). Most IEEE journals/conferences use
% top floats exclusively. 
% Note that, LaTeX2e, unlike IEEE journals/conferences, places
% footnotes above bottom floats. This can be corrected via the
% \fnbelowfloat command of the stfloats package.




%\section{Conclusion}
%The conclusion goes here.




% conference papers do not normally have an appendix


% use section* for acknowledgment
%\section*{Acknowledgment}
%
%
%The authors would like to thank...





% trigger a \newpage just before the given reference
% number - used to balance the columns on the last page
% adjust value as needed - may need to be readjusted if
% the document is modified later
%\IEEEtriggeratref{8}
% The "triggered" command can be changed if desired:
%\IEEEtriggercmd{\enlargethispage{-5in}}

% references section

% can use a bibliography generated by BibTeX as a .bbl file
% BibTeX documentation can be easily obtained at:
% http://mirror.ctan.org/biblio/bibtex/contrib/doc/
% The IEEEtran BibTeX style support page is at:
% http://www.michaelshell.org/tex/ieeetran/bibtex/
\bibliographystyle{IEEEtran}
% argument is your BibTeX string definitions and bibliography database(s)
\bibliography{AspectsBib}
%
% <OR> manually copy in the resultant .bbl file
% set second argument of \begin to the number of references
% (used to reserve space for the reference number labels box)

%\begin{thebibliography}{19}
%%
%%%\bibitem{IEEEhowto:kopka}
%%%H.~Kopka and P.~W. Daly, \emph{A Guide to \LaTeX}, 3rd~ed.\hskip 1em plus
%%%  0.5em minus 0.4em\relax Harlow, England: Addison-Wesley, 1999.
%%
%\bibitem{1Review} A Review of Requirement Engineering Issues and Challenges in Various Software Development Methods
%\bibitem{2Pohl}	Pohl, 2010; 
%\bibitem{3Eckhardt}	Jonas Eckhardt. “Categorizations of Product-related Requirements in Practice Observations and Improvements”
%\bibitem{4Robertson}	Robertson and Robertson,2012; 
%\bibitem{5Sommerville} Sommerville and Kotonya, 1998;
%\bibitem{6Lamsweerde}	Van Lamsweerde, A. (2001). Goal-oriented requirements engineering: A guided tour. In
%Proceedings of the 5th International Symposium on Require-ments Engineering (RE),
%\bibitem{7DO-178C}	DO-178C.
%\bibitem{8DO-331}	DO-331
%\bibitem{9ISO29148}	ISO29148:2011
%\bibitem{10Broy}	 Broy, M. (2016). Rethinking Nonfunctional Software Requirements: A Novel Approach Categorizing System and Software Requirements. In Hinchey, M., editor, Software Technology: 10 Years of Innovation in IEEE Computer. John Wiley \& Sons/IEEE Press.
%\bibitem{11Mager}	 Mager, P. (2015). Towards a Profound Understanding of Non-Functional Requirements. Master’s thesis, Tech-nische Universität München.
%\bibitem{12RQS}	Requirement Quality Suite 
%\bibitem{13EARS}	EARS
%\bibitem{14UML}	UML 
%\bibitem{15Goals}	“Does Goal-Oriented Requirements Engineering Achieve its Goal?”, Tuefl, Eckardt, Mund, Femmer
%\bibitem{16NaPiRe}	Supporting Requirements-Engineering Research That Industry Needs. The NaPiRE Initiative” by Daniel Mén-dez Fernández
%\bibitem{17MiniDuide}	Mini-Guideline to Requirements Engineering. Birgit Penzenstadler
%\bibitem{18MIRA}	 MIRA framework
%\bibitem{4citation} B.H.C. Cheng and J.M. Atlee, “Research Directions in Requirements Engineering,” Proc. 2007 Future of Software Eng. (FOSE 07), 2007, pp.285–303.
%
%
%\end{thebibliography}

% that's all folks
\end{document}


