\section{Evaluation and Validation}
Firstly, categorization-based process has been validated by industry use case within ASSET project during avionics component development. A method for guiding requirements review based on aspects was elaborated. 
%The outcames of this  – what impact the Aspects make on requirements reviewing process – handle requirements complexity.

The second step in V\&V process of categorization-based approach is to investigate within inner Fortiss project (``Beyond AF3”), if aspects concept is general enough to ensure  in-dependability from use of concrete tools. The main idea of this use case is to apply the aspects concept within other than AF3 tool i.e. GitLab, and explore, if the concept holds up its main characteristics independently from AF3 tool.

Next, for evaluation of the categorization-based concept a series of empirical study can be carried out. Such studies can be conducted within Fortiss research institute to find out how a user can handle problems of requirements quality applying \textit{aspects}.

Though, according to article about real industry needs in requirements-engineering area: ``It’s not surprising that the biggest challenge in RE research is to provide proper empirical figures that demonstrate particular success factors. However, those factors are critical determinants of what works in practice and what doesn’t.\cite{4citation} Consequences are that the state of empirical evidence in RE is particularly weak and that much of everyday industrial practices is governed by conventional wisdom rather than empirical evidence. \cite{16NaPiRe}''This is a further motivation to investigate the question of an appropriate and adequate validation techniques for ``aspects'' concept in this study.