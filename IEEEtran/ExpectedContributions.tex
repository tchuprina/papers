\section{ExpectedContributions}
Conducting this research the following goals will be achieved:
\begin{itemize}
\item	to bridge a gap between requirements engineers and reviewers providing to them a feasible method based on requirements \textit{aspects}, which guides both roles within requirements reviewing process and supports a communication flow between these two roles. That leads to ease of redundancy in the work-flow,
\item	to close a gap between requirements reviewer and system developers by providing a structured constitution of requirements which is in help to concentrate a precise overview on the requirements (as on a whole scope of them so as only on specific category of requirements or part of them, depending on categorization granularity). As a result, a reviewer can provide a more specified and detailed response to developers on any changes in a system,
\item	moreover, \textit{aspects} concept links a bridge between requirements engineers and developers, so that a structure of \textit{aspects} itself includes key “hints” for every requirements category, which help in seamlessly mapping requirements into design of a system,
\item	the \textit{aspects} is a general concept, which can be applied independently on a specific tool chain. They can be simply applied with a familiar development tool. It gives a flexibility in use and reduces the cost for new tools learning,
\item	semi-automatically check-lists, which are an extensions for every aspect, useful for making a requirements review easier and, consequently, less time- and resource-consuming.
\end{itemize}
All in all, the \textit{aspects} concept provides a common instrument for all mentioned above roles (requirements engineers, reviewers, developers). All these points can increase a quality of requirements and thereby, save a significant amount of resources in system development such as project time and expenses.
