
\section{Related Work}

The scientific base of our research includes study work about requirements categorization from doctoral thesis of Jonas Eckhardt,2017 \cite{3Eckhardt}. There the author considers general requirements categorizations commonly applied in industry and recognized by the scientific community (i.e. Pohl, 2010; Robertson and Robertson,2012; Sommerville and Kotonya, 1998; Van Lamsweerde, 2001 )\cite{2Pohl},\cite{4Robertson},\cite{5Sommerville},\cite{6Lamsweerde} and investigates requirements categorization based on a system model and its impact on requirements quality, relying on state-of-the-Art research results such as Broy 2016\cite{10Broy}, Mager 2015 \cite{11Mager} and others. 

The next paper, which made an impact to start this research was an article \cite{15Goals}, where the authors raise doubts about applying of goal-oriented requirements engineering (GORE) in practice due to missing sufficient documentation for industry and lack of comprehension between the scientific society and industry. A consequence paper \cite{16NaPiRe} describes current problems in industry with concern to requirements engineering field and possible reasons for existing gap between researchers and industry.

The industry perspective of the study of categorization-based concept embraces practical knowledge assembled by industry partners of Fortiss with conformity to industry standards and regulations such as DO-178C \cite{7DO-178C}, DO-331 \cite{8DO-331}, ISO29148:2011 \cite{9ISO29148} etc., which comprise guidelines and recommendations for system development process.

\textit{Currently, the industry operates with requirements in the form of text (use cases, specifications with detailed lines of description or bulleted list, etc.), usually poor structured and error-prone resulting in inferior quality of requirements and the problems associated with this, which described above. Most popular tools in industry for requirements structuring and to handle traceability include DOORS\cite{DOORS}, Requirement Quality Suite (RQS) \cite{12RQS}, BTC. Alternatively, in practice patterns can be applied to bring the requirements into formalized view, by using languages such as EARS \cite{13EARS} or UML\cite{14UML} etc. However, this approach doesn't give flexibility in terms of requirements engineering and provides no guidance for a user (requirements engineer, reviewer or developer). 

In contrary to such techniques, the categorization-based method provides a noticeable degree of freedom for requirements engineering. It allows to work with unstructured or unorganized textual requirements and adhere incrementally a “skeleton” based on requirements categories. Furthermore, in contrast to general categorizations advocated by many researchers, the categorization-based concept supports a method for elaborating requirements and reviewing them. The categorization-based concept incarnates a core of common categorizations and go far ahead guiding a process for requirements engineering activities in a project. This approach provides a possibility for a user to compose a set of appropriate categories with different granularity of any category in the set.

Moreover, the idea of the categorization-based process is general and allows a user to be independent from specific tools. It can be applied within already known for a user chain of tools and thereby, save costs and time for a project realization instead of training employees of new extra tools. 
}
%Additionally, the mentioned above tool examples have on a hunch an assumption, that input for the designing phase should be brought to a completed structure before developers start their job. Such implication doesn’t look realistic due to changeability of requirements during time-line of a project and could cause developers to be involved in requirements engineering performance. It can lead to risks rise, difficulties in standards compliance and reflect on costs of a project. The Aspects concept allows to avoid such problems and change requirements during a whole system life cycle.
