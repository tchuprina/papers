\section{Current Status}
At the moment, the research goes further with the idea of \cbc. It implies to consider an impact of \cc on a whole development process including testing phase. Also the study is supposed to cover requirements reuse and a role of the \cbc in supporting that process.

Currently, question of relevant and sufficient methods for proper evaluation of the \cbc still arises. The investigation of possible techniques is in process.


%considers relevant and sufficient methods for proper evaluation of the \cbc. Simultaneously to this research task, further implementation of the \asp concept in \af tool and its testing continues. Moreover, the study is supposed to consider an impact of the \cbc on other phases of development process, such as system testing, requirements reuse and etc.

%Next steps within the research appear as a checking of usability of the concept conducting a series of empirical experiments.


%Implemented within the MIRA framework \cite{18MIRA} in \autof and reflecting requirements engineering activities form the initial stage of a development process, the \textit{aspects} can represent a general concept supporting its unique method, which may be applied within other tools thereby, providing tool\textbackslash platform- independence. The next pace in this direction of the research is a deeper investigation of this theory and its consideration on practice.
%
%And the most challenging question about proper and adequate validation technics for the Aspects concept will be considered in future work in this study.
