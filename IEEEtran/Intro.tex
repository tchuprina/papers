\section{Introduction}
\label{sec:intro} 

%problem: - RE doesn't want to use goal-oriented approaches [ref.''does GORE achieve its goal?''], advocated in RE society; patterns or formal languages as UML - doesn't give such a freedom as CARE [ref: Patterns in the Requirements Engineering: A Survey and Analysis Study - ``One can reuse pattern knowledge to solve recurring problems,'']

According to a review on requirement engineering~\cite{1Review}, challenges regarding quality of requirements, communication gap between teams in development process bring about impediments to system development and increase consume of resources in industry, such as time and costs. Moreover, research on industry needs~\cite{16NaPiRe} provides statistics of failed projects caused by the mentioned problems. Unstructured, poor written textual requirements can be a reason for the described challenges. 
``At the beginning of the RE process, mainly natural language (informal representations) is used to document the requirements for the system'' [Klaus Pohl and Nelufar Ulfat-Bunyadi]

Scientific society advocates Goal-Oriented Requirements Engineering (GORE) as a possibly suitable solution for elaboration of well-organized requirements. However, despite of a acceptance of the GORE in academia, the idea doesn't spread so broadly in industry. [Does GORE achieves its goal?]. Another, a well-know and recognized by industry approach for structuring textual requirements is patterns and formal languages, likewise UML and EARS[ref: Patterns in the Requirements Engineering: A Survey and Analysis Study]. Forming requirements by set of rules and constraints, the approach confines requirements engineers in work with complex, unstructured requirements.

In this paper we introduce \care (CARE), that aims at handling unorganized, textual requirements categorizing them by their characteristics, and supporting a work flow in requirements engineering among all stakeholders. \ca method brings unformalized, vague textual requirements into 
a structured form applying general categorization to them; so that, requirements engineer (RE) 
proceeds with text of the requirements without re-writing them into formal sentences. 
In other words, \ca doesn't force RE to formalize requirements with help of rigid patterns or 
formal languages, but leaves a freedom in creativity with elaboration of textual requirements; 
on the other hand, \ca structures the requirements for other stakeholder as a system designer or a reviewer.
\ca organize requirements, written in natural language, into a structure based on their categories (functional and non-functional).

Requirements consists of incoming information to all levels of an implemented system. 
We can consider these inputs as meta-data of the requirements. Thereby, \ca structures 
requirements' meta-data by categories building a ``framework'' of textual requirements. 
This framework with the input-data constitutes a \textit{new artifact} in system development process. 
It represents requirements as a meta-model attached to the textual requirements.

Moreover, as long as requirements meta-data spread throughout all system development phases 
(system defining, design, testing and others), all stakeholders in system development life-cycle 
can employ \care to contribute into requirements engineering process. Consequently, we affirm, that 
\ca implies also a method, that supports categorization-based process flow for all stakeholders, 
rather than just an approach for requirements categorization. In this paper, we make an example of \ca usage, 
implemented in \autof.

Summing up all points from above, we postulate, that \care impacts efficiently on requirements 
engineering process in system development. It allows to work with unstructured textual requirements 
categorizing them and to analyze such requirements more precisely. Additionally, \ca provides a common 
instrument for all stakeholder roles within system development life-cycle, that maintains mutual understanding
between a development team. As a consequence, quality of textual requirements increases. This results in saving up
project resources, such as time-consume for system development and implementation costs.


%provide a description of \cbc (CBC), that suggests a solution for handling the problems with concern to the quality of system requirements. \cc brings textual unstructured requirements to a consistent view categorizing them by their characteristics. Moreover, \cc supports a method for transforming requirements into less complex composition and for reviewing them in a more precise way. By that means,\cbc improves qualitative aspect of requirements.

%This concept embraces general requirements categorization advocated in the scientific society and includes experience gathered by the industry in software-intensive systems development. On practice, incoming document for system developers can be system requirements in the form of text (use cases, specifications with detailed lines of description or bulleted list, etc.), which are usually poor structured and error-prone. The list of most popular requirements management tools in the industry community include DOORS~\cite{DOORS}, Requirement Quality Suite (RQS)~\cite{12RQS}, BTC. Alternatively, to bring the requirements into formalized view, system designers can apply patterns by using languages such as EARS~\cite{13EARS} or UML~\cite{14UML} etc. %However, this approach doesn't give flexibility in terms of requirements engineering and provides no guidance for a user  (requirements engineer, reviewer or developer). 

%In contrary, the \cbc categorization-based provides a method with noticeable degree of freedom from any particular tools for requirements engineering and reviewing process. It allows to work with unstructured or unorganized textual requirements by adhering incrementally categories to them. 

\cbc has been incarnated as a framework within \autof and named \asp. Here \asp represent requirements categories with regards to their characteristics. In \af  the \asp have been implemented as templates supporting main facets of the requirements. For instance, \textit{Signal aspect} categories the requirement, which provides a signal definition. This \textit{aspect} captures information about the signal type and range in the template. More detailed description of \asp as an instrument of \cc provides further in \autoref{sec:cbcinaf}. 

\subsection{Motivation}
\label{sec:Motivation} 
Motivation for this research is to elaborate and investigate \cbc for structuring textual requirements, that helps in solving the problems regarding requirements quality. Additionally, to study an impact of category-based approach on requirements development process. This motivation leads to the following research questions: 

\textbf{Research question 1:} \textit{How to measure a quality of requirements?}

\textbf{Research question 2:} \textit{How to develop a set of requirements, which corresponds to requirements of high level quality?}

Evaluation of a proposed method in requirements engineering field is a challenging task, which requires an effort from a researcher due to its subjective nature~\cite{16NaPiRe}. In this study raises the question of appropriate and adequate evaluation techniques and methods for the concept assessment, which allow to demonstrate the categorization-based concept feasibility on practice. The research will cover estimation of requirements quality with outcomes comparison before and after application of the categorization-based concept.

Based on these questions we have formulated these hypotheses:

\textbf{Hypothesis 1:} The \cbc for structuring textual requirements can improve their quality. The \cc appears as a general concept, which provides independence from a certain chain of tools. It grants a user flexibility in selecting categories with an appropriate level of granularity. 

\textbf{Hypothesis 2:} The \cbc for structuring textual requirements serves to support requirements engineering process and procedure of review. It provides a common instrument for all roles of system development process. It maintains a traceability between requirements on all stages of system design. The concept works with requirements meta-data seamlessly integrating it through the whole development process and thereby, improves quality of requirements.

%Importance of requirements quality extremely rises for safety-critical systems. An example is an avionics domain, where risk of error should be minimized from the initial stage of development process, during requirements elicitation and analysis. Otherwise, the flaw can affect a system design on different phases of the development process and results in crucial consequences.
%The categorization-based concept incarnates a core of common categorizations and go far ahead guiding a process for requirements engineering activities in a project. This approach provides a possibility for a user to compose a set of appropriate categories with different granularity of any category in the set.
%
%Moreover, the idea of the categorization-based process is general and allows a user to be independent from specific tools. It can be applied within already known for a user chain of tools and thereby, save costs and time for a project realization instead of training employees of new extra tools. 
%
%
%Requirements are categorized by their characteristics. Therefore, one or more categories can be attached to a single requirement, depending on the requirement's composition. Number of applied categories increases with rising complexity of the considered requirement. In this case, requirement with a range of categories should be splitted into several simplified requirements with less number of attached categories, if it's possible with respect to context.
%
%User can define, which categories correspond to every requirement. Moreover, \cbc provides a certain degree of freedom for user to choose a granularity of every attached category. In other words, user makes a decision, how detailed the requirements should be considered and categorized. After analysis of whole scope of requirements, every requirement should be categorized.  
%
%Output of this procedure is structured and more consistent requirements with attached set of corresponding categories. This set constitutes a new artifact in system development process, that represents a full structure of the requirements. Such structured requirements are more precise and unambiguous. Thanks for that, reviewing process becomes easier and more accurate; risk of mistakes in system design declines due to better legibility of requirements. We can conclude, that quality of requirements increases with applying \cbc.
%
%
%The idea behind categorization-based concept arises from general requirements categorization and goes ahead with the arrangement of requirements in formal structure categorizing them by their characteristics and thereby, forming an attached combination of the categories (\textit{aspects}). This combination of \textit{aspects} reflects a structure of the requirements scope and constitutes a new Artifact.  
