\section{Problems and Motivation}
In software-intense systems development there are many types of obstacles and challenges, which still make an influence on quality and cost of designed systems as well as on a development process flow remaining unsolved or not proper considered.
 
According to information from a review of requirement engineering issues and challenges in various software development methods \cite{1Review}, challenges regarding requirements quality, communication gap between roles included in development process such as requirements engineer, reviewer or software developer etc., lack of structured process of mapping requirements to design (Semi-Automatic Process of Generic Template Creation) bring about plenty impediments into software development process of various kinds and may result in great increase of resources from industry. Moreover, a research of industry needs\cite{16NaPiRe} provides a statistics of failure projects also causing by the mentioned problems among another, pointed the most critical issues such as ``incomplete or hidden requirements'', ``communication flaws within the project team'', ``separation of requirements from solutions'', ``time boxing'', ``moving target''\cite{16NaPiRe}. All these points can lead to risks rise, difficulties in standards compliance and reflect on costs of a project.
%All these points show a demand of requirements engineering as a complete field of study.
 
Therefore, motivation for this research is to elaborate and investigate a categorization-based concept for structuring textual requirements, that helps in solving the mentioned problems regarding requirements quality. Additionally, to study an impact of category-based approach on requirements development process. This motivation leads to the following research questions: 

\textbf{Research question 1:} \textit{How to measure a quality of requirements?}

\textbf{Research question 2:} \textit{How to develop a set of requirements, which corresponds to requirements of high level quality?}

Evaluation of a proposed method in requirements engineering field is a challenging task, which requires an effort from a researcher due to its subjective nature \cite{16NaPiRe}. In this study raises the question of appropriate and adequate evaluation techniques and methods for the concept assessment, which allow to demonstrate the categorization-based concept feasibility on practice. The research will cover estimation of requirements quality with outcomes comparison before and after application of the categorization-based concept.

Based on these questions we have formulated these hypotheses:

\textbf{Hypothesis 1:} The categorization-based concept for structuring textual requirements can improve their quality regardless of a system domain or a specific chain of tools. It allows a user to select only necessary categories with an appropriate level of granularity forming a new artifact, which shapes a meta-data of requirements.

\textbf{Hypothesis 2:} The categorization-based concept for structuring textual requirements serves in support of requirements engineering process providing a common instrument for all roles of this process. It maintains a traceability between requirements on all stages of system design. The concept works with requirements meta-data seamlessly integrating that through the whole development process, and thereby makes an impact on improvement of requirements quality.

Importance of requirements quality extremely rises for safety-critical systems. An example is an avionics domain, where risk of error should be minimized from the initial stage of development process, during requirements elicitation and analysis. Otherwise, the flaw can affect a system design on different phases of the development process and results in crucial consequences.






