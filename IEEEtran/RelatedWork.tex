\section{Related Work}
\label{sec:relatedwork} 
%Related work and contrebution w.r.t. literature.
The scientific base of our research includes study work about requirements categorization~\cite{3Eckhardt}. There the author considers general requirements categorizations commonly applied in industry and recognized by the scientific community~\cite{2Pohl},~\cite{4Robertson},~\cite{5Sommerville},~\cite{6Lamsweerde} and investigates requirements categorization based on a system model and its impact on requirements quality, relying on state-of-the-Art research results such as~\cite{10Broy},~\cite{11Mager} and others. 

The next paper, which made an impact to start this research was~\cite{15Goals}, where the authors raise doubts about applying of goal-oriented requirements engineering (GORE) in practice due to missing sufficient documentation for industry and lack of comprehension between the scientific society and industry. A consequence paper~\cite{16NaPiRe} describes current problems in industry with concern to requirements engineering field and possible reasons for existing gap between researchers and industry.

From the industry perspective, \cbc study embraces practical knowledge assembled by industry partners of Fortiss with conformity to industry standards and regulations, such as DO-178C~\cite{7DO-178C}, DO-331~\cite{8DO-331}, ISO29148:2011~\cite{9ISO29148} etc. The standards comprise guidelines and recommendations for system development process.

Our research contributes to the requirements categorization topic, investigating an influence of the general \cbc on requirements quality.

Commonly recognized in scientific community, the goal-based requirements categorization doesn't seem to prevalent in the industry. The study of \cbc will expose the idea, which goes ahead of general requirements categorization, providing a common instrument within system development process. Moreover, this concept will bring a method, that guides requirements engineering and review activities. Our research will grant the approach for improving quality of requirements by their structuring using \cbc. 

% to save a significant amount of resources in system development, such as project time and expenses.


%Currently, the industry operates with requirements in the form of text (use cases, specifications with detailed lines of description or bulleted list, etc.), usually poor structured and error-prone resulting in inferior quality of requirements and the problems associated with this, which described above. Most popular tools in industry for requirements structuring and to handle traceability include DOORS\cite{DOORS}, Requirement Quality Suite (RQS) \cite{12RQS}, BTC. Alternatively, in practice patterns can be applied to bring the requirements into formalized view, by using languages such as EARS \cite{13EARS} or UML\cite{14UML} etc. However, this approach doesn't give flexibility in terms of requirements engineering and provides no guidance for a user (requirements engineer, reviewer or developer). 

%In contrary to such techniques, the categorization-based method provides a noticeable degree of freedom for requirements engineering. It allows to work with unstructured or unorganized textual requirements and adhere incrementally a “skeleton” based on  requirements categories. Furthermore, in contrast to general categorizations advocated by many researchers, the categorization-based concept supports a method for elaborating requirements and reviewing them. The categorization-based concept incarnates a core of common categorizations and go far ahead guiding a process for requirements engineering activities in a project. This approach provides a possibility for a user to compose a set of appropriate categories with different granularity of any category in the set.

%Moreover, the idea of the categorization-based process is general and allows a user to be independent from specific tools. It can be applied within already known for a user chain of tools and thereby, save costs and time for a project realization instead of training employees of new extra tools. 

%Additionally, the mentioned above tool examples have on a hunch an assumption, that input for the designing phase should be brought to a completed structure before developers start their job. Such implication doesn’t look realistic due to changeability of requirements during time-line of a project and could cause developers to be involved in requirements engineering performance. It can lead to risks rise, difficulties in standards compliance and reflect on costs of a project. The Aspects concept allows to avoid such problems and change requirements during a whole system life cycle.
