\section{Conclusion}
\label{sec:conclusion}
The presented metrics has its pros and cons. The advantages of this approach described further. 

Firstly, the proposed metrics provide a uniformity in measurements of requirements quality. They set a distinct baseline for comparison of requirements quality between different projects. The quantitative assessment provides an unambiguous characteristic of quality, taking into account maturity of the requirements, time spent for RE and development process.

The next advantage is, it is a simplicity of the provided metrics in comparison with some of the considered in \autoref{sec:relatedwork} methods. 

Additionally, the described metrics can be used in empirical measurements to assess the quality of requirements only after project completion.

On the other hand, we have to mention also problems of this approach. The provided metrics take in to account time for RE and development processes and amount of requirements changes, but doesn't consider the quality of project. Another point is, that the approach has not been yet applied at practice. However, we consider it to apply in our study regarding requirements categorization approach.

%\tatiana{Pros:} 
%\begin{itemize}
	%\item The proposed metrics provide a uniformity in measurements of requirements quality; 
	%\item baseline for all assessment; 
	%\item Simplicity; 
	%\item can be used in empirical measurements.
%\end{itemize}


%\tatiana{Cons:}
%\begin{itemize}
	%\item  Firstly, the problem of this approach is: it doesn't take into account the quality of project and its correlation with the requirements quality.
%\item Secondly, the approach is theoretic (academic) and has not yet been applied on practice.
%\end{itemize}