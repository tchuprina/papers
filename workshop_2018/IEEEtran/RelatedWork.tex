\section{Related Work}
\label{sec:relatedwork} 

\subsection{what's quality means?}

Despite on multiple publications about requirements quality and its assessment, the ``quality'' as a term is still subjective.~\cite{Mund:2017},~\cite{Femmer:2017}.
Industry standards [ISO/IEC/IEEE:29148:201] prescribe characteristics and criteria, which presumed effective for improving quality of requirements e.g completeness, unambiguity for sets of requirements and difficulty, dependency for a single requirement. Moreover, Lamsweerde provides possible defect-based checklist~\cite{Lamsweerde:2009}.
Now exists a vast majority of quality definitions:
\begin{enumerate}
	\item in industry Standards - set of attributes such as completeness, unambiguity... [ISO/IEC/IEEE-29148];
	\item in scientific point of view, [Lamsweerde] gives a general list of characteristics, [Pohl] proposed a framework defining dimensions of quality (• the specification dimension, • the representation dimension, the agreement dimension.) ;
	\item check requirements language for lacking of errors, defects, ambiguity and possible reasons for incomprehension [];
	\item different kind of quality models such as activity-based[10], natural-language requirements specifications[7] %(activity, stupid!)
\end{enumerate}


\subsection{Metrics for quality measure}
\begin{itemize}
	\item methodology for context-specific RE artifact quality measuring has in its base an activity-based quality model [activity, stupid!] - brief description
	\item the mentioned attributes should be satisfied 
	\item some researchers shift their look to product quality measurement e.g [measuring success], however 
\end{itemize}

All these metrics intend to intrinsic inspection of requirements. In contrast to them, we propose the quantitative metrics for assessment of requirements quality