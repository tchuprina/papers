\section{Related Work}
\label{sec:relatedwork} 

\subsection{What does ``quality'' mean?}

Despite on multiple publications about requirements quality and its assessment, 
the term ``quality'' is still subjective.~\cite{Mund:2017},~\cite{Femmer:2017}. 
Industry standards~\cite{ISO/IEC/IEEE:2011} specify characteristics and criteria, 
which presumed effective for improving the quality e.g completeness, unambiguity and others. 
Additionally, researchers provide several types of quality definition and methods for its assessment. 
In example, Lamsweerde provides a defect-based checklist to inspect requirements for possible flaws 
and errors in ~\cite{Lamsweerde:2009}; Pohl proposes a framework defining dimensions of quality: 
the specification dimension, the representation dimension, the agreement dimension~\cite{POHL:1994}. 
Another approach implies syntactic check of the requirements text for improving its comprehension, 
correctness, ambiguity and other akin characteristics e.g.~\cite{Ferrari:2014},~\cite{Berry:2006}.
All these metrics intend to intrinsic inspection of requirements. In comparison with them, activity-based 
quality models shift their approach from inherent properties to a context of process, and represent 
a meta quality model~\cite{Wagner:2012},~\cite{Femmer:2015}. Furthermore, the "quality question" also turns 
to a consideration of how the requirements quality impacts to project success and the relation between 
them in scientific surrounding~\cite{Emam:1995},~\cite{Kamata:2007}, so as among practitioners~\cite{BeattyHokanson:2014}.

The presented in this paper metrics consider generally requirements artifact for its maturity and a process 
of adjusting the requirements at RE and implementation phases, in a quantitative way, a posteriori.


%Now exists a vast majority of quality definitions:
%\begin{enumerate}
	%\item in industry Standards - set of attributes such as completeness, unambiguity... [ISO/IEC/IEEE-29148];
	%\item in scientific point of view, [Lamsweerde] gives a general list of characteristics, [Pohl] proposed a framework defining dimensions of quality (• the specification dimension, • the representation dimension, the agreement dimension.);
	%\item check requirements language for lacking of errors, defects, ambiguity and possible reasons for incomprehension [];
	%\item different kind of quality models such as activity-based[10], natural-language requirements specifications[7] %(activity, stupid!)
%\end{enumerate}
%
%
%\subsection{Metrics for quality measure}
%\begin{itemize}
	%\item methodology for context-specific RE artifact quality measuring has in its base an activity-based quality model [activity, stupid!] - brief description
	%\item the mentioned attributes should be satisfied 
	%\item some researchers shift their look to product quality measurement e.g [measuring success], however 
%\end{itemize}
%
%All these metrics intend to intrinsic inspection of requirements. In contrast to them, we propose the quantitative metrics for assessment of requirements quality