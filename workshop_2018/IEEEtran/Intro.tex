\section{Introduction}
\label{sec:Intro} 

\subsection{Problem}
Still the question about quality of requirements remains problematic~\cite{Fernandez2016}. 
``How to measure the quality of requirements?'' is a subjective question, because 
there is a few quantitative metrics for measuring the quality of requirements. 
All of them are looking on intrinsic words of requirements and, therefore, depend on their statement.

\subsection{Contribution}
Here we present various quantitative metrics for assessing the quality of requirements based 
assuming a relation between the requirements quality and changes of the requirements. Comparing with existing approaches, 
discussed in section \autoref{sec:relatedwork}, our method considers a relation between quality of requirements and 
the resulted product measuring number of changes and time-consumption during RE and SD phases. Here we consider 
the corrections in requirements document done within requirements engineering (RE) and software development (SD) stages~\cite{FARBEY:1990}, 
and their influence on the time for development process. The suggested metric takes into account a maturity of the requirements 
and reflects its leverage on the product, resulting in a number from 0 (bad) - 1 (good) for a quality assessment. A developed system, 
which has passed an acceptance test by a customer, is considered as a baseline for a resulting product. That stands, the proposed metric 
for requirements quality can be applied after a project completion and serves as a method for assessment in research works. 

The presented approach is planned for measuring quality of requirements in our current study regarding requirements 
categorization and its impact on a system development life cycle.
