\section{Introduction}
\label{sec:Intro} 
Initial phase of system development process (Requirements engineering) orchestrates a whole work flow of the development. 
It initiates with collecting customers' requirements, which are objectives for a future system; it continues with their  analyses 
and transformations into high level and low level requirements, which are inputs for system design process; 
and finally, it goes with testing the system design against the requirements. It means, that requirements serve as inputs to a system 
development and influence the result of this process; apparently, their quality correlate with the quality of the ending product [].

Still the question about measuring quality of requirements remains problematic due to its subjectivity[]. However, the assesment 
of the requirements of quality can lead to a variety of activities in the requirements of analyses and, cosequently, support 
a more precise culculation of required resources such as, time for project realization.

Here we present our mathematical approach for assessing a quality of requirements based on assumption about 
a relation between the requirements quality and a quality of the resulting product.Comparering with existing approaches, 
discussed in section \autoref{sec:relatedwork}, our method considers a relation between quality of requirements and 
the resulted product measuring number of changes and time-consumption during RE and SE phases. Here we consider 
the corrections in requirements document done within requirements engineering (RE) and software enineering (SE) stages[Farbey,1990], 
and their influence on the time consuming for development process. The suggested metric takes into account a maturity of the requirements 
and reflects its leverage on the product, resulting in a number from 0 (bad) - 1 (good) for a quality assessement. A developed system, 
which has passed an acceptance test by a customer, considered as an ethalon of a resulting product for a measure of its quality. 
Here succeded project means a good quality of a released product, and a failed project refers to a bad quality of the product. 
That stands, the proposed metric for requirements quality can be applied after a project completion and serves as an assessment method 
for research works. The presented approach is planned for measuring quality of requirements in our current study regarding requriements 
categorization and its impact on a system development life cycle.
