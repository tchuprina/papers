\section{Introduction}
\label{sec:Intro} 

\subsection{Problem}
The question about quality of requirements remains problematic~\cite{Fernandez2016}: 
``How to measure the quality of requirements?'' is a subjective question.  
There are only few quantitative metrics to measure the quality of requirements. 
All of them are looking at intrinsic characteristic of requirements and, therefore, depend on their statement.

\subsection{Contribution}
We present various quantitative metrics for assessing the quality of requirements assuming a relation 
between the requirements quality and changes of the requirements. Comparing with existing approaches, 
discussed in section \autoref{sec:relatedwork}, our method considers a relation between quality of 
requirements and the resulted product measuring number of changes and time-consumption during RE and 
implementation phases. We consider the corrections in requirements document done within 
requirements engineering (RE) and implementation stages~\cite{FARBEY:1990}, and their influence on the time 
for development process. The suggested metrics take into account a maturity of the requirements 
and reflects its leverage on the product, resulting in a number from 0 (bad) - 1 (good) for a quality assessment.  
A developed system, which has passed an acceptance test by a customer, is considered as a baseline 
for the resulting product. Importantly, the proposed metrics are usable to assess the quality of requirements 
only after project completion. 

The presented approach can be considered for empirical studies. The presented metrics are planed for measuring 
the quality of requirements in our current study regarding requirements categorization.
