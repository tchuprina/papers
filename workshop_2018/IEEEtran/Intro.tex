\section{Introduction}
\label{sec:Intro} 

\subsection{Problem}
Measuring the quality of requirements remains problematic~\cite{Fernandez2016} due to its subjectivity.
There are only few quantitative metrics to measure the quality of requirements. 
All of them are looking at intrinsic characteristic of requirements, e.g., lack of syntactic defects 
or conformity to qualitative attributes. Such approach makes them dependable on the requirements' statement and, 
therefore, decreases an accuracy of the measurement. In other words, the clear baseline for the comparison of results 
from several separate projects is vague. Assessment of same quality characteristics within different project don't assume equality in outcoming criteria.  


\subsection{Contribution}
We present various quantitative metrics for assessing the quality of requirements assuming a relation 
between the requirements quality and corrections of the requirements done within RE and system implementation work flow. 
Comparing with existing approaches, discussed in \autoref{sec:relatedwork}, our method considers the quality of 
requirements with respect to the process, measuring the number of changes and time-consumption during RE and 
implementation phases. We consider the changes in requirements document done within 
requirements engineering and implementation stages~\cite{FARBEY:1990}, and their influence on the time 
for development process. The suggested metrics take into account a maturity of the requirements 
and reflects its leverage on the product, resulting in a number from 0 (bad) - 1 (good) for a quality assessment.  
A developed system, which has passed an acceptance test by a customer, is considered as a baseline 
for the resulting product. Importantly, the proposed metrics are usable to assess the quality of requirements 
only after project completion. 

The presented approach can be considered for empirical studies; and is in plan to employ in our study 
for doctoral thesis regarding requirements categorization approach. The requirements quality with appliance of the requirements categorization approach will be measured by the proposed metrics; afterwards, the results will be compared with an outcome of the project without the requirements categorization approach.