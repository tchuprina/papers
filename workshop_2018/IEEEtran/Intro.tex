\section{Introduction}
\label{sec:Intro} 
Initial phase of system development process (Requirements engineering) orchestrates a whole work flow of the development. It initiates with collecting customers' requirements, which are objectives for a future system; it continues with their  analyses and transformations into high level and low level requirements, which are inputs for system design process; and finally, it goes with testing the system design against the requirements. It means, that requirements serve as an income to a system development and influence the result of this process; apparently, their quality correlate with the quality of the end product [].

Still, the question about measuring quality of requirements remains problematical due to its subjectivity[].
\tatiana{discover a problem here}

 Here we present our approach to the quality measurement suggesting its mathematical explanation. Comparering with existing approaches, discussed in section , ours (title) considers a relation between quality of requirements and the resulted product measuring time and resources consumption for requirements engineering phase and further software development.

quality of requirements

resulted product

time and resources consumption 