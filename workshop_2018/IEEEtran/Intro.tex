\section{Introduction}
\label{sec:Intro} 

\subsection{Problem}
Measuring the quality of requirements remains problematic~\cite{Fernandez2016},~\cite{Mund:2017} 
due to subjectivity of quality definition. There are only few quantitative metrics to measure the quality of requirements. 
All of them are looking at intrinsic characteristic of requirements, e.g., whether the requirement’s statement 
includes any syntactic defects or conforms to qualitative attributes such as, unambiguity, completeness or consistency. 
Such approach depends on the requirements' statement and, therefore, decreases an accuracy of the measurement. 
In other words, the clear baseline for the comparison of the requirements' quality is vague. 
Thereby, proposing the quantitative metrics for measuring requirements' quality, we attempt 
to solve the problem with the subjective evaluation of quality and provide a uniform and more precise measure. 
 
For example, the identical initial set of ``raw'' requirements is an input to a project for a team of model-based developers as well as to a group of employees from another company;% to assess the quality of the same requirements'; %estimating the requirements' quality
The output of the requirements evaluation from these two groups may vary depending on the methods used for quality estimation and approaches applied for development. That means, the estimator (a person, who measures the quality) faces with the problem of comparison of the quality estimations.  

Our quantitative metrics is an effective instrument to compare the results of the requirements' quality estimation applying different techniques, within different companies.
%Evaluating same quality characteristics of the similar requirements within different projects do not assume equality in outcoming criteria.  


\subsection{Contribution}
We present various quantitative metrics for assessing the quality of requirements assuming a relation 
between the requirements quality and the number of requirements' corrections to be done within RE and system implementation workflow. 
Comparing with existing approaches, discussed in \autoref{sec:relatedwork}, our method considers the quality of 
requirements with respect to the process, measuring the number of changes and time-consumption during RE and 
implementation phases. We consider the changes in requirements document done within 
requirements engineering and implementation stages~\cite{FARBEY:1990}, and their influence on the development time, 
instead of analyzing the internal characteristics of the requirements' statements. 
The suggested metrics take into account the maturity of the requirements, 
resulting in a number from 0 (bad) - 1 (good) for a quality assessment.  
A developed system, which has passed an acceptance test by a customer, is considered as a baseline 
for the resulting product. Importantly, the proposed metrics are usable to assess the quality of requirements 
only after project completion, since this quality is based on the final requirements.

The presented approach can be considered for empirical studies; we plan to employ it in our study 
for doctoral thesis. % to analyze an influence of requirements' categorization on quality aspect. 
% The requirements quality with appliance of the requirements categorization approach will be measured by the proposed metrics; afterwards, the results will be compared with an outcome of the project without the requirements categorization approach.