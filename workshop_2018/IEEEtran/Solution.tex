\section{Metrics}
\label{sec:Solution} 
%
%\begin{figure}[H]
	%\centering
		%\includegraphics[width=0.8\textwidth,height=2in]{MetricsDiag.pdf}
	%\caption{Requirements engineering and development processes}
	%\label{fig:Metrics_shot}
%\end{figure}

%In these metrics we consider a maturity of requirements as a characteristic of the quality.
%
\autoref{fig:Metrics_shot} provides a graphical explanation of the metrics. The workflow is divided into requirements engineering (RE) 
and system implementation (SI) processes' iterations along the time line: $t_{1},t_{2},...,t_{i}$ - 
instants of time for RE phases (RE iteration); $t_{1}',t_{2}',...,t_{i}'$ - 
instants of time for implementation process based on the provided requirements (SI iteration).
The output of every RE iteration is RE artifact (depicted as $R_{1},R_{2},...,R_{i}$); and every SI phase results into SI artifact, 
e.g., software architecture (shown as SW), till the end of the SI phase, which leads to the release of a product and end of the project.  

Consequently, to calculate the \textit{maturity index} of the requirements, the following metrics should be determined:

 \begin{equation}\label{eqn:mu1}
\mu_{1}(R_{j}) = i-j
	\end{equation}
	
\textrm{end of the number of iterations for the requirements} $R_{j}$;

 \begin{equation}\label{eqn:mu2}
\mu_{2}(R_{j}) = t_{i}-t_{j}    
 \end{equation}

\textrm{amount of time (in hours) between initial and the last phases of RE process applying the requirements} $R_{j}$;

%\begin{equation}\label{eqn:mu3}
%\mu_{3}(R_{j}) = \displaystyle\sum_{j} t_{j+1}-t_{j}\acute{}
%\end{equation}
%
%,\textrm{total amount of time (in hours) required for SI process};


%(here, along with requirements' maturity, we can also talk about such quality criterion as the requirements' comprehensibility);
%$\mu_{4} = \displaystyle\sum_{j} (t_{j+1}-t_{j}\acute{} + p_{j}*(t_{j}\acute{} - t_{j}))$

\newpage
\hfill \break

\vspace{5.4cm}

Every RE artifact (a document with requirements) has its index of maturity. The range of this index varies from 0 to 1: 0 means ``bad'' and 1 indicates ``good'' quality.
The maturity index is inferred from a number of iterations (a certain amount of changes applied to requirements' document) and the time spent for RE and implementation process.
The more mature a requirement is, the less changes (iterations) the artifact requires, and the shorter the time for the development process. 
\textsl{Thus, the better quality of the requirements: the higher index of the requirements maturity} (\autoref{eqn:maturity_ind}). 

In \autoref{eqn:maturity_ind}, the maturity index indicates, how far the considered requirements with the certain maturity parameters (presented as a sum of the metrics for the RE artifact - $\sum\mu_{n}(R_{j})$) from their ``good'' state (defined as 1)

  \begin{equation}\label{eqn:maturity_ind}
\textrm{\textit{maturity index}} = \frac{1}{\sum\mu_{n}(R_{j})}
	\end{equation}

where $(R_{j}$) is a considered RE artifact; $j$ is an initial iteration for calculation.
%where $\mu(R_{j})$ is a calculated number of the iterations for a considered RE artifact $(R_{j}$); $j$ is an initial iteration for calculation.

The input for the first iteration of RE process is a "raw" requirements. During the RE phase, the requirements become elaborated (changed with respect to a project's demands); the result of this process is a corrected RE artifact, that is an input to the next stage (SI). Therefore, we consider the initial RE artifact as a document \textit{$\mu(R_{j})=0$}. 


